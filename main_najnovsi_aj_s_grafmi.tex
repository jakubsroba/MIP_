% main.tex — AITP: Umelá inteligencia a trh práce – predbežná verzia projektu
% Tím LITTLETON (FIIT STU), 1. sem 2025/26


\documentclass[12pt,a4paper]{article}
\usepackage[utf8]{inputenc}
\usepackage[T1]{fontenc}
\usepackage[slovak]{babel}
\usepackage{lmodern}
\usepackage{geometry}
\usepackage{graphicx}
\usepackage{caption}
\geometry{margin=2.5cm}
\setlength{\parskip}{0.6em}
\setlength{\parindent}{0pt}
\graphicspath{{./img/}} % <- sem uložte všetky PNG grafy

\title{LITTLETON: Analýza dopadu umelej inteligencie na trh práce v Európskej únii}
\author{%
Jakub Šroba \and
Samuel Škombár \and
Bruno Stehlík \and
Adam Sklenčík
}
\date{\today}

\begin{document}
\maketitle

\begin{center}
\textbf{Akronym projektu: LITTLETON}
\end{center}
\begin{abstract}
Rozvoj umelej inteligencie (AI) vytvára nové pracovné príležitosti, zároveň však zrýchľuje
automatizáciu rutinných činností a zmeny v štruktúre dopytu po zručnostiach. Tento dokument
predstavuje \emph{predbežnú verziu} projektu, ktorého cieľom je zostrojiť otvorený analytický
pipeline pre zber, čistenie a porovnávaciu analýzu dát o pracovných ponukách a štatistikách
trhu práce v krajinách EÚ. Budeme kombinovať API ponúk práce (Adzuna) s oficiálnymi
štatistikami (Eurostat, OECD) a vizualizovať trendy rozvoja AI-profesií, podielu ICT
špecialistov a indikátorov automatizácie. Výstupom bude interaktívny dataset a grafy
(Power BI/Plotly) vhodné pre študentov, výskumníkov a tvorcov politík.
\end{abstract}
\section*{Grafy}

\begin{figure}[h]
  \centering
  \includegraphics[width=\linewidth]{automation_risk_HIGH_share_by_industry.png}
  \caption{Podiel ponúk s vysokým automatizačným rizikom (HIGH) podľa odvetvia.}
\end{figure}

\begin{figure}[h]
  \centering
  \includegraphics[width=\linewidth]{eu_latest_top10.png}
  \caption{Top 10 krajín EÚ podľa najnovšieho podielu AI pracovných ponúk.}
\end{figure}

\begin{figure}[h]
  \centering
  \includegraphics[width=\linewidth]{eu_leaders_trends.png}
  \caption{Trend: Top krajiny EÚ podľa najnovšej hodnoty podielu AI pracovných ponúk.}
\end{figure}

\begin{figure}[h]
  \centering
  \includegraphics[width=\linewidth]{eu_mean_timeseries.png}
  \caption{Podiel AI pracovných ponúk v EÚ (priemer krajín) v čase.}
\end{figure}

\begin{figure}[h]
  \centering
  \includegraphics[width=\linewidth]{eu_mean_timeseries_explained.png}
  \caption{Podiel AI pracovných ponúk v EÚ (priemer krajín) — s vyznačenými udalosťami a dôvodmi zmien trendu.}
\end{figure}

\begin{figure}[h]
  \centering
  \includegraphics[width=\linewidth]{eu_vs_noneu_timeseries.png}
  \caption{EÚ-27 vs.\ Non-EU — porovnanie podielu AI pracovných ponúk v čase.}
\end{figure}

\begin{figure}[h]
  \centering
  \includegraphics[width=\linewidth]{industry_weighted_automation_risk_rank.png}
  \caption{Riziko automatizácie podľa odvetvia (vážené: High + $0{.}5 \cdot$ Medium).}
\end{figure}

\begin{figure}[h]
  \centering
  \includegraphics[width=\linewidth]{skills_weighted_risk_rank.png}
  \caption{Riziko automatizácie podľa skillu (vážené: High + $0{.}5 \cdot$ Medium).}
\end{figure}

\end{document}