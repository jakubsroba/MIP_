\documentclass[12pt,a4paper]{article}
\usepackage[utf8]{inputenc}
\usepackage[T1]{fontenc}
\usepackage[slovak]{babel}
\usepackage{lmodern}
\usepackage{geometry}
\usepackage{graphicx}
\usepackage{hyperref}
\geometry{margin=2.5cm}
\setlength{\parskip}{0.6em}
\setlength{\parindent}{0pt}
\graphicspath{{./img/}} % <- sem uložte všetky PNG grafy


\title{Analýza dopadu umelej inteligencie na trh práce v Európskej únii\thanks{Semestrálny projekt v predmete Metódy inžinierskej práce, ak. rok 2025/2026, vedenie: Ing. Richard Marko, PhD.}} % meno a priezvisko vyučujúceho na cvičeniach

\author{%
Jakub Šroba\textsuperscript{1} \and
Samuel Škombár\textsuperscript{1} \and
Bruno Stehlík\textsuperscript{1} \and
Adam Sklenčík\textsuperscript{1}
\vspace{8pt} \\ % Medzera pred inštitúciou
\textsuperscript{1}{\small Slovenská technická univerzita v Bratislave}\\
{\small Fakulta informatiky a informačných technológií}\\
{\small \texttt{\{xsroba, xskombarm, xstehlik, xsklencik\}@stuba.sk}} % Použitie spoločného email formátu
}
\date{\today}

\begin{document}
\maketitle

\begin{center}
\textbf{Akronym projektu: LITTLETON}
\end{center}

\begin{abstract}
Rozvoj umelej inteligencie (AI) vytvára nové pracovné príležitosti, zároveň však zrýchľuje
automatizáciu rutinných činností a zmeny v štruktúre dopytu po zručnostiach. Tento dokument
predstavuje \emph{predbežnú verziu} projektu, ktorého cieľom je zostrojiť otvorený analytický
pipeline pre zber, čistenie a porovnávaciu analýzu dát o pracovných ponukách a štatistikách
trhu práce v krajinách EÚ. Budeme kombinovať API ponúk práce (Adzuna) s oficiálnymi
štatistikami (Eurostat, OECD) a vizualizovať trendy rozvoja AI-profesií, podielu ICT
špecialistov a indikátorov automatizácie. Výstupom bude interaktívny dataset a grafy
(Power BI/Plotly) vhodné pre študentov, výskumníkov a tvorcov politík.
\end{abstract}

\section{Úvod}

Umelá inteligencia (neskôr AI) patrí medzi najvýznamnejšie technologické inovácie súčasnosti.
Jej vývoj sa zrýchľuje a čoraz viac preniká do rôznych oblastí života – od priemyslu až po školstvo.
Vďaka schopnosti spracúvať veľké množstvo dát a samostatne sa rozhodovať mení spôsob, ako pracujeme.
Naším cieľom je analyzovať dostupné dáta a následne ich vizualizovať.
Taktiež chceme zvýšiť povedomie bežnej populácie o štatistikách týkajúcich sa AI
a motivovať ich k tomu, aby sa učili moderným technológiám.

\section{Metodika}

Analýza bola založená na zostavení dátového súboru obsahujúceho celkový počet pracovných miest,
počet pozícií súvisiacich s umelou inteligenciou a ich podiel na zamestnanosti pre jednotlivé krajiny a roky.
Dáta pre dátový súbor sme použili z viacerých zdrojov ako: \cite{kaggle-ai-impact-2024-2030, opendatabay-aiml, hiringlab-ai-tracker}.
Následne boli pracovné pozície rozdelené do kategórií podľa typu činnosti – AI roly a bežné roly,
pričom bežné pozície boli ďalej členené na administratívne, logistické a výrobné.
Po klasifikácii sa uskutočnila agregácia údajov podľa počtu pozícií súvisiacich s umelou inteligenciou,
čím sa získali hodnoty pre jednotlivé sektory.

Tieto údaje boli následne spojené s makroekonomickými dátami,
najmä s ukazovateľmi podielu ICT špecialistov,
aby bolo možné skúmať vzťahy medzi rozvojom AI a úrovňou digitalizácie pracovnej sily.

\section{Dopad AI na pracovné miesta}

Z našich doterajších zistení vyplýva, že dopyt po AI stále rastie
a nové pracovné ponuky pribúdajú, čo môžete vidieť vizualizované v nasledujúcich grafoch.

\begin{center}
    \includegraphics[width=\linewidth]{eu_latest_top10.png}
\end{center}

\begin{center}
    \includegraphics[width=\linewidth]{eu_leaders_trends.png}
\end{center}

Dáta boli porovnané aj pre krajiny EÚ a mimo EÚ.
Na základe týchto dát vieme povedať, že krajiny v EÚ majú mierne väčší podiel AI ponúk
ako krajiny, ktoré v EÚ nie sú.
\cite{ilo}Tento trend podporuje štúdia ILO, ktorá porovnáva krajiny G20 a tie, ktoré v G20 nie sú.
Na základe dát vieme povedať, že bohatšie krajiny majú výrazne viac ponúk zameraných na AI.

\begin{center}
    \includegraphics[width=\linewidth]{eu_vs_noneu_timeseries.png}
\end{center}

\section{Čo je v dnešnej dobe dôležité?}

Štúdia ILO zdôrazňuje, že kľúčovým faktorom úspešného prispôsobenia sa novým technológiám
sú zručnosti pracovníkov.
Najviac sa uplatňujú kombinácie digitálnych, kognitívnych a sociálno-emocionálnych schopností,
ako je tímová spolupráca, komunikácia či riešenie problémov.
Taktiež zvýrazňuje potrebu rozvíjať \emph{soft skills},
ktoré umožňujú efektívne pracovať, využívať technológie,
prispôsobovať sa zmenám a kreatívne riešiť úlohy.

\section{Ďalšie Grafy}

\begin{center}
    \includegraphics[width=\linewidth]{automation_risk_HIGH_share_by_industry.png}
\end{center}

\begin{center}
    \includegraphics[width=\linewidth]{industry_weighted_automation_risk_rank.png}
\end{center}

\begin{center}
    \includegraphics[width=\linewidth]{skills_weighted_risk_rank.png}
\end{center}

\section{Diskusia}
V buducnosti sa projekt zameria na zber dát o pracovných ponukách a trendoch AI, napríklad LinkedIn, Github, Eurostat a Udemy \cite{linkedin-api, github-api, eurostat-api, udemy-api}.
Každé API poskytuje čiastočne odlišné údaje, čo umožňuje porovnanie a overenie kvality dát.
Dáta sa dajú normalizovať a spojiť do jednotného datasetu, čím vznikne komplexnejší obraz trhu práce.
Modulárna štruktúra pipeline umožňuje pridávať nové API bez zásadných zmien v analýze.
Takéto prepojenie viacerých zdrojov zvyšuje presnosť predikcií a podporuje lepšie rozhodovanie o rozvoji zručností pracovníkov.


\bibliographystyle{plain}
\bibliography{sources}
\end{document}
