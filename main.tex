% main.tex — Predbežná verzia MIP (AI a trh práce v EÚ)
\documentclass[11pt,a4paper]{article}
\usepackage[T1]{fontenc}
\usepackage[utf8]{inputenc}
\usepackage[slovak]{babel}
\usepackage{amsmath, amssymb}
\usepackage{graphicx}
\usepackage{booktabs}
\usepackage{siunitx}
\usepackage[colorlinks=true, allcolors=blue]{hyperref}
\usepackage{csquotes}
\usepackage{geometry}
\geometry{margin=2.5cm}
\usepackage{url}

\title{Analýza dopadu umelej inteligencie na trh práce v Európskej únii}
\author{%
Jakub Šroba \texttt{xsroba@stuba.sk} \and
Samuel Škombár \texttt{xskombar@stuba.sk} \and
Bruno Stehlík \texttt{xstehlik@stuba.sk} \and
Adam Sklenčík \texttt{xsklencik@stuba.sk}
}
\date{\today}

\begin{document}
\maketitle

\begin{abstract}
TOTO JE KONCEPT:\newline
Rozvoj umelej inteligencie (AI) vytvára nové pracovné príležitosti, zároveň však zrýchľuje
automatizáciu rutinných činností a zmeny v štruktúre dopytu po zručnostiach. Tento dokument
predstavuje \emph{predbežnú verziu} projektu, ktorého cieľom je zostrojiť otvorený analytický
pipeline pre zber, čistenie a porovnávaciu analýzu dát o pracovných ponukách a štatistikách
trhu práce v krajinách EÚ. Budeme kombinovať API ponúk práce (Adzuna) s oficiálnymi
štatistikami (Eurostat, OECD) a vizualizovať trendy rozvoja AI-profesií, podielu ICT
špecialistov a indikátorov automatizácie. Výstupom bude interaktívny dataset a grafy
(Power BI/Plotly) vhodné pre študentov, výskumníkov a tvorcov politík.
\end{abstract}

\section{Úvod}
Vplyv AI na trh práce je predmetom intenzívnej diskusie. Popri náraste dopytu po profesionáloch
v oblasti strojového učenia a dátovej vedy sa očakáva \cite{FreyOsborne2017,AcemogluRestrepo2018,OECD2019}
aj preskupovanie pracovných miest v dôsledku automatizácie, najmä v rutinných a opakovateľných
profesiách. Cieľom tohto projektu je empiricky zmapovať \emph{vývoj ponúk práce súvisiacich s AI}
a \emph{kontextové ukazovatele} (napr. podiel ICT špecialistov) v rámci EÚ a poskytnúť
prehľadné vizualizácie trendov.

\section{Ciele a výskumné otázky}
\subsection*{Hlavný cieľ}
Vytvoriť dátový pipeline a vizualizačný portál, ktorý umožní porovnať vývoj AI a non-AI
pracovných ponúk v EÚ v rokoch 2023–2025 a zasadiť ich do kontextu štatistík trhu práce.
\subsection*{Výskumné otázky}
\begin{enumerate}
  \item Ako sa mení počet a podiel \textit{AI-príbuzných} pracovných ponúk medzi krajinami EÚ a v čase?
  \item Existuje súvis medzi intenzitou AI ponúk a podielom ICT špecialistov v krajine?
  \item Líšia sa trendy medzi sektormi/profesijnými skupinami?
  \item Aké limity a skreslenia vyplývajú z dostupných dát a definícií?
\end{enumerate}

\section{Súvisiaca literatúra a zdroje}
Budeme vychádzať z výskumu o technologických zmenách a trhu práce
\cite{FreyOsborne2017,AcemogluRestrepo2018,WFoJ2023,OECD2019,ILO2024}
a z oficiálnych zdrojov dát (Eurostat, OECD) a otvorených pracovných portálov (Adzuna API)
\cite{AdzunaAPI,EurostatDB,OECDData}.

\section{Dáta a zber dát}
\paragraph{Pracovné ponuky (Adzuna API).}
Získame ponuky pre vybrané krajiny EÚ a obdobie 2023–2025. Pre každú ponuku sa uložia minimálne:
názov, popis, lokácia, dátum, firma (ak je), kategória/odvetvie. Predspracovaním vytvoríme indikátor
\texttt{is\_AI} podľa pravidiel v~\ref{sec:klasifikacia}.

\paragraph{Oficiálne štatistiky.}
Z~Eurostatu načítame ukazovatele týkajúce sa zamestnanosti a podielu ICT špecialistov (percento na
zamestnanosti). Z~OECD použijeme indikátory súvisiace s automatizáciou a digitalizáciou (napr. index
rizika automatizácie podľa profesií). Konkrétne identifikátory datasetov doplníme po validácii.

\section{Spracovanie a metodika}
\subsection{ETL pipeline}
\begin{enumerate}
  \item \textbf{Extrakcia:} API dotazy (Adzuna) a sťahovanie CSV z~Eurostat/OECD.
  \item \textbf{Transformácia:} čistenie textov, normalizácia krajín (ISO kódy), jednotné dátumy;
        prevod JSON$\rightarrow$tabuľka; agregácie podľa krajiny a mesiaca.
  \item \textbf{Načítanie:} ukladanie do jednotných CSV/Parquet; notebook \texttt{AI\_Trh\_Prace.ipynb}
        sprístupňuje kroky a vizualizácie.
\end{enumerate}

\subsection{Klasifikácia AI vs. non-AI}\label{sec:klasifikacia}
Pracovná definícia: ponuka je \emph{AI-príbuzná}, ak \emph{názov alebo popis} obsahuje výraz
z kurátorskej množiny kľúčových slov (napr. „machine learning“, „artificial intelligence“,
„deep learning“, „LLM“, „computer vision“, „MLOps“) alebo je v AI-príbuznej kategórii.
Kľúčové slová budú lokalizované (EN/SK/CZ/DE) a manuálne kontrolované.
Výstupom je binárny príznak \texttt{is\_AI}.

\subsection{Metriky a hypotézy}
\paragraph{Metriky.}
Medzimesačná a medziročná miera rastu, podiel AI ponúk na všetkých ponukách,
korelácia s podielom ICT špecialistov, index intenzity AI ponúk na \SI{100000}{obyv.}

\paragraph{Hypotézy.}
\begin{itemize}
  \item H1: Podiel AI ponúk rastie v období 2023–2025 vo väčšine krajín EÚ.
  \item H2: Krajiny s vyšším podielom ICT špecialistov vykazujú vyššiu intenzitu AI ponúk.
  \item H3: Trendy sa líšia podľa sektorov; najvyšší rast v ICT a pokročilých výrobách.
\end{itemize}

\section{Priebežné výsledky}
V tejto fáze máme pripravený návrh dátového pipeline (obr.~\ref{fig:pipeline})
a pracovný notebook v Colab/Jupyteri (\texttt{AI\_Trh\_Prace.ipynb}), ktorý obsahuje
prvé dotazy na API a načítanie ukážkových datasetov. Po dokončení zberu dát doplníme
grafy trendov a porovnania medzi krajinami.

\begin{figure}[h]
  \centering
  \includegraphics[width=0.9\linewidth]{figures/diagram_pipeline.png}
  \caption{Návrh spracovania dát (ETL $\rightarrow$ analýza $\rightarrow$ vizualizácie).}
  \label{fig:pipeline}
\end{figure}

\section{Harmonogram a pracovné balíky}
\begin{table}[h]
\centering
\begin{tabular}{llp{7.5cm}}
\toprule
\textbf{Balík} & \textbf{Termín} & \textbf{Výstup} \\
\midrule
PB1: ETL skeleton & Týždeň 1–2 & Adzuna fetcher, Eurostat/OECD loader, jednotné CSV \\
PB2: Klasifikácia AI & Týždeň 2–3 & Kurátorské kľúčové slová, validácia na vzorke 200 ponúk \\
PB3: Analýza trendov & Týždeň 3–4 & AI vs. non-AI časové rady, medziročné zmeny \\
PB4: Vizualizácie & Týždeň 4–5 & Power BI/Plotly dashboard (krajiny, mesiace, sektory) \\
PB5: Článok \& formy & priebežne & Rozšírenie kapitol, citácie, VV-A..VV-F \& PDF \\
\bottomrule
\end{tabular}
\end{table}

\section{Riziká, limity a etické aspekty}
\textbf{Pokrytie dát:} API pracovných ponúk je skreslené (duplicitné inzeráty, selekcia portálov).
\textbf{Definícia „AI“:} pravidlové rozpoznanie môže mať falošné pozitíva/negatíva; plánovaná
manuálna kontrola vzorky. \textbf{Porovnateľnosť krajín:} rozdielne zvyklosti v inzerátoch
a kategóriách. \textbf{Etika:} pracujeme s verejne dostupnými inzerátmi; osobné údaje
sa neukladajú (anonymizácia identifikátorov).

\section{Záver}
Predstavili sme plán a priebežné výstupy pre analýzu AI na trhu práce v krajinách EÚ.
Nasleduje finalizácia zberu dát, klasifikácie ponúk a príprava dashboardu.
Priebežne rozširujeme prehľadový článok a dopĺňame citácie.

\bibliographystyle{ieeetr}
\bibliography{refs}
\end{document}
